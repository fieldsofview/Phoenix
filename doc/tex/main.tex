\documentclass[a4paper,12pt]{report}
\usepackage[tmargin=1in,bmargin=1in,lmargin=1in,rmargin=1in]{geometry}
\usepackage{titlesec}
\usepackage{xcolor}
\usepackage{PTSans}
\usepackage{droidserif}
\usepackage[T1]{fontenc}
\usepackage{fontspec}
\defaultfontfeatures{Ligatures=TeX}

% Set sans serif font to PT Sans
%\setsansfont{\usefont{T1}{PTSans-TLF}{b}{n}PT Sans Bol}
% Set serifed font to Droid Serif
%\setmainfont{Droid Serif}
% Set formats for each heading level
\titleformat*{\section}{\fontsize{16}{18}\bfseries\sffamily}
\titleformat*{\subsection}{\fontsize{14}{16}\bfseries\sffamily}
\titleformat*{\subsubsection}{\fontsize{12}{14}\bfseries\sffamily}
\begin{document}

\title{Phoenix Documentation}

\include{title}
 
\chapter{Introduction}
Phoenix is a distributed agent-based simulation platform designed for
development of agent-based systems and simulations. We use agent based
models and simulations in our study of complex systems,
micro-simulations and bottom-up approaches to engineering.

This document explains the motivations, design, architecture of the
new agent based simulation framework. It will also explain the
motivations behind some of the design decisions so that we provide the
right hooks for people looking to extend or modify the platform for
themselves. 

The platform is released with both the source code and the design
documents as well as adequate explanations.

This documentation will also guide users to develop simple agent-based
models and also include some case-studies of some models as well as
possible ideas for extension of the platform.

\chapter{Motivation}
In this section we discuss the motivation to create a new agent based
 platform.
\section{Modelling Complex Systems}
There are many approaches to modelling complex systems. We would like
to classify them broadly into predictive models and generative
models. (The generative models are similar to the approach to
modelling as described by Epstine et.al in there book, generative
social science.) The nature of complex systems do not allow for
building of predictive models. Complex systems interact in intractable
ways which are not easily succeptable to data observations. Thus, a
lot of data and observations will not be avaliable to build and
compare the built models.

\subsection{An example to expalin complex system behaviour}

Let us take a simple thought experiment to imagine complex
systems. Imagine looking at a film-reel. A film reel is a series of
photographs that are taken one after another forming a series of
photographs. Each photograph forms a frame. By playing the reel at 30
frames per second gives us the illusion of moving picture, a movie
(due to persistence of vision).

Linear systems are like a film reel. By following the frames on the
reel one can deduce the story that the pictures are saying. In case a
frame or a set of frames are missing in between them you can always
see the frames at the extremities and \textit{draw} new frames or
deduce to a good extent, what the pictures in the missing frames would
look like. This is because the film real follows a linear causality,
i.e. it is ensured that after one frame with one picture, the next
frame will hold the next logical frame. Given this guarantee one can
go back in the reel to deduce lost pictures. One can also draw new
pictures, at the end of a reel. We will also be able to tell where a
scene began by tracing our steps back to the using the frames as the
frames remain static. Thus, a linear, static, tractable system.

Now imagine instead of a single strip of film you are given a fabric
of such photograph frames (as if on a checker pattern blanket where
each box has a frame). To begin with how would one start reading such
a piece? There is no definite start or end as any part of such a
blanket can become the start or the end. There is another
complication; imagine if the pictures do not have a causal relation
ship, i.e. the adjacent pictures have nothing in common. There is no
define way to establish a story. Given a starting point we have no
idea as to which picture to jump next. This complicates the matter
immensely as one cannot justify where the movie would start or
end. One also fails to fill missing frames as we do not know how the
frames are related. If a person starts filling frames randomly, there
is no way for us to say if the scene is correct or, to make it worse,
how did the scene even come about in the first place. In this blanket
example, there are only two dimensions of frames, but, what if there
were more dimensions? Thus, the system becomes intractable and
non-linear. To add to our woes if the frames also kept shifting their
places then the system becomes dynamic too!


Another reason for non-avaliability of data is that most complex
system components are themselves areas of active research. For example
behavioural economics still conducts experiments on how a community
considers utility of various resources at its disposal.

However, one can still study complex systems using generative
models. The objective of such models is in their ability in explaining
the behaviour of the system i.e. it can explain the trajectory of
certain outcomes of the complex system. This infomration can be used
to understand or look for failure (or unwanted) states and thus test
designs for such complex systems.
\section{Complex Systems and Agent-Based Models}
\section{Other approaches to studying complex systems}
\subsection{Systems Dynamics}
\subsection{Complex Networks}

\chapter{Design and Architecture of Phoenix}
%\chapter{Design and Architecture}
The framework provides infrastructure to create agents, agent
behaviour, messaging, output and logging broadly forming the:

\begin{itemize}
\item Agent Definition Mechanism
\item Communication Layer
\item Visualisation or Output Module
\end{itemize}

The main components of Phoenix are:
\begin{itemize}
\item Agent Controller
\item Agents
\item Message Queues
\item Output Module
\item Logging
\end{itemize}

\section{Agent Controller}
\textit{AgentController}(AC) is a container within which agents live
and execute their behaviours. ACs communicate using AMQP protocol. An
AC is responsible for the following:

\begin{itemize}
\item Agent creation
\item Inter agent messaging
\item Inter AC messaging.
\item Writing simulation output.
\item Gracefully shutting-down the simulation locally. 
\end{itemize}

\section{Agents}
An agent represents a real world or abstract entity which is a vital
actor in the simulation. Agents have a list of attributes which define
the agent and their uniqueness with respect to other agents of the
same type.

\subsection{ Agent Attributes }
Agent attributes is a abstract class. For a particular type of Agent
(e.g. Person) the class has to to extended and required set of
attributes defined. The values of the attributes could be stored in a
database or in a XML file.

\subsection{ Agent Behaviour }
Agents have objectives, beliefs and preferences or biases by which
they exist in the system. To achieve its objective an agent has to
undertake certain actions to manipulate its data, interact with other
agents etc, this is done through actions calls behaviours. An agent
may have one or more behaviours behaviour which are executed in a
defined order.

\section{ Messaging }
The messaging subsystem (or communication channel) is a vital
component in a distributed system which facilitates information
sharing and synchronisation. In Phoenix the communication is entirely
confined to ACs. Our messaging system had to satisfy the following
requirements:

\begin{itemize}
\item High throughput - A few thousand messages should be delivered,
  end-to-end, in a second.
\item High availability - The system should not crash under high load.
\item Asynchronous messaging - Synchronous messaging will slow down a
  large system. A asynchronous technique which works like mail boxes
  is desired.
\item Platform independence - It should not be tied to a particular
  programming language or a operating system, this will enable us to
  build heterogeneous ACs.
\end{itemize}

\textit{RabbitMQ} is our choice of the message queuing system. It is
based on the Advanced Message Queuing Messaging Protocol (AMQP)
standard and is open source under Mozilla Public License. RabbitMQ has
interfaces in languages such as Java, .Net, Python and others and
supports multiple schemes of communication including 1-to-1,
1-to-many, Store-and-forward, file-streaming and others. Messages are
transmitted in the form of binary data and hence any form of
encryption/decryption has to be implemented at the client side.

\section{ Output System }
This is work in progress. Currently only \textit{log4j} logs are
available for output analysis. We plan to include a graphing system.

\section{ Logging }
The framework uses \textit{log4j}, a java based logging utility under
Apache License, Version 2.0 to provide detailed logging of all system
level actions by the various entities.


\chapter{Implementation of Phoenix}
%\chapter{Implementation}
The framework is implemented in Java. Java provides ease of
programming and has a wide range of helpful libraries. In this section
the code is explained to some extent.

\section{ AgentController }
\textit{AgentController} is an abstract class. The
\textit{AgentController} is responsible for running agents, writing
the output, communicating with other ACs and so on.  \textit{<NEED TO
  EXPLAIN FURTHER>}

\subsection{ Agent }
Agents are implemented as Java threads. Agent is an abstract class
with at least the following attributes:

\begin{itemize}
\item \textit{AID} - a unique identifier for the agent.
\item \textit{objectiveFlag} - indicates whether the agent has
  achieved its objective.
\item \textit{statusFlag} - a true value indicates that the agent has
  finished his task for the current tick, and a false value indicates
  that the agent hasn't finished execution.
\item \textit{compositeBehaviour} - one or more behaviours which the
  agent will run depending on the agents preferences and objectives.
\end{itemize}

The framework enables the definition of several types of agents, and
the creation of a large number of agents of each type.

Every agent will have a set of behaviours, and choose to exhibit(run)
a particular behaviour based on its internal logic. Behaviours are
implemented using the composite design pattern. Behaviour is an
abstract class with a method called 'run', which receives all the
parameters of the agent in an object of type
\textit{AgentAttributes}. Every agent type will have its specific
behaviour implementation, for example \textit{PersonMoveBehaviour} (to
move a person agent), \textit{VehicleMoveBehaviour} etc.

The parameter (\textit{AgentAttributes}) defines the current state of
the agent.  For example, a person agent will have attributes like
health, speed, curiosity etc. \textit{AgentAttributes} is itself an
abstract class, and each agent type will have its own implementation
of this class. \textit{Person} agent type will have
\textit{personAttributes}, \textit{Vehicle} will have
\textit{VehicleAttributes} and so on.

Every agent contains a 'run' method. While the implementation of this
method, other attributes of the agent, its preferences and behaviours
depend on the simulation to be run, the logic of the agent will be
internalised within this method.

\section{ Messaging }
Each AC runs in its own JVM and inter AC communication is through the
\textit{RabbitMQ} message queue. Every AC has an input queue, which is
identified by the host IP address and a queue name. To communicate
with another AC, a AC has to write the message to the recipients
message queue. The AC is notified as and when a new message is
received through a listener implemented in a method called
\textit{receivedMessage(Message)}.

\subsection{ Message }
A message is a has the following fields:
\begin{itemize}
\item type - is an integer and identifies the type of message
\item content - is an Integer to indicate the status of the sending AC
\item sender - is a string and identifies the sender (is the host name
  of the sender)
\end{itemize}

\subsection{ QueueManager }
Manages the delivery of a message to the recipient and listening to
incoming messages.
\begin{itemize}
\item queueUser - the AC using this queue.
\item queueParameters - parameters of the input queue for the AC
\end{itemize}

\subsubsection{ QueueUser }
This is an interface using which any class can receive messages on an
input queue. Any class which wants to receive and send messages should
implement this interface. \textit{QueueUser} enforces the 'Observer'
design pattern on the implementing class. Every AC (which implements
\textit{QueueUser}) registers with a \textit{QueueManager}, and
whenever a message is received \textit{QueueManager} notifies the AC
by calling the \textit{receivedMessage(Message)} method of the AC.

\subsubsection{ QueueParameters }
This defines the the parameters of the input queue.
\begin{itemize}
\item \textit{queueName} - name of the queue.
\item \textit{username} - username for RabbitMQ 
\item \textit{password} - password for RabbitMQ
\end{itemize}

\subsection{ Database Module }
\textit{<NEED TO COMPLETE THIS>}

\section{ High Level Diagrams}
%{{agentStructure.png | General Structure of An Agent}}
%{{typesOfUniverse.png | Types of universes that an agent can occupy}}
%{{structureOfAC.png | Structure of an Agent Controller}}
%{{structureOfACNetwork.png | Structure of an Agent Controller Network}}
%{{flowOfMessages.png | Flow of Messages}}
%{{highLevelArch.png | High Level Architecture of Phoenix}}


\chapter{Extending Phoenix and Module Architecture}
\section{Current Modules}
\section{Future Modules}

\chapter{Using Phoenix}
\section{Basic Configuration}
\section{High Avaliability Configuration}

\chapter{Performance of Phoenix}

\chapter{Example Models using Phoenix}

\end{document}
