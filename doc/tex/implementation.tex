%\chapter{Implementation}
The framework is implemented in Java. Java provides ease of
programming and has a wide range of helpful libraries. In this section
the code is explained to some extent.

\section{ AgentController }
\textit{AgentController} is an abstract class. The
\textit{AgentController} is responsible for running agents, writing
the output, communicating with other ACs and so on.  \textit{<NEED TO
  EXPLAIN FURTHER>}

\subsection{ Agent }
Agents are implemented as Java threads. Agent is an abstract class
with at least the following attributes:

\begin{itemize}
\item \textit{AID} - a unique identifier for the agent.
\item \textit{objectiveFlag} - indicates whether the agent has
  achieved its objective.
\item \textit{statusFlag} - a true value indicates that the agent has
  finished his task for the current tick, and a false value indicates
  that the agent hasn't finished execution.
\item \textit{compositeBehaviour} - one or more behaviours which the
  agent will run depending on the agents preferences and objectives.
\end{itemize}

The framework enables the definition of several types of agents, and
the creation of a large number of agents of each type.

Every agent will have a set of behaviours, and choose to exhibit(run)
a particular behaviour based on its internal logic. Behaviours are
implemented using the composite design pattern. Behaviour is an
abstract class with a method called 'run', which receives all the
parameters of the agent in an object of type
\textit{AgentAttributes}. Every agent type will have its specific
behaviour implementation, for example \textit{PersonMoveBehaviour} (to
move a person agent), \textit{VehicleMoveBehaviour} etc.

The parameter (\textit{AgentAttributes}) defines the current state of
the agent.  For example, a person agent will have attributes like
health, speed, curiosity etc. \textit{AgentAttributes} is itself an
abstract class, and each agent type will have its own implementation
of this class. \textit{Person} agent type will have
\textit{personAttributes}, \textit{Vehicle} will have
\textit{VehicleAttributes} and so on.

Every agent contains a 'run' method. While the implementation of this
method, other attributes of the agent, its preferences and behaviours
depend on the simulation to be run, the logic of the agent will be
internalised within this method.

\section{ Messaging }
Each AC runs in its own JVM and inter AC communication is through the
\textit{RabbitMQ} message queue. Every AC has an input queue, which is
identified by the host IP address and a queue name. To communicate
with another AC, a AC has to write the message to the recipients
message queue. The AC is notified as and when a new message is
received through a listener implemented in a method called
\textit{receivedMessage(Message)}.

\subsection{ Message }
A message is a has the following fields:
\begin{itemize}
\item type - is an integer and identifies the type of message
\item content - is an Integer to indicate the status of the sending AC
\item sender - is a string and identifies the sender (is the host name
  of the sender)
\end{itemize}

\subsection{ QueueManager }
Manages the delivery of a message to the recipient and listening to
incoming messages.
\begin{itemize}
\item queueUser - the AC using this queue.
\item queueParameters - parameters of the input queue for the AC
\end{itemize}

\subsubsection{ QueueUser }
This is an interface using which any class can receive messages on an
input queue. Any class which wants to receive and send messages should
implement this interface. \textit{QueueUser} enforces the 'Observer'
design pattern on the implementing class. Every AC (which implements
\textit{QueueUser}) registers with a \textit{QueueManager}, and
whenever a message is received \textit{QueueManager} notifies the AC
by calling the \textit{receivedMessage(Message)} method of the AC.

\subsubsection{ QueueParameters }
This defines the the parameters of the input queue.
\begin{itemize}
\item \textit{queueName} - name of the queue.
\item \textit{username} - username for RabbitMQ 
\item \textit{password} - password for RabbitMQ
\end{itemize}

\subsection{ Database Module }
\textit{<NEED TO COMPLETE THIS>}

\section{ High Level Diagrams}
%{{agentStructure.png | General Structure of An Agent}}
%{{typesOfUniverse.png | Types of universes that an agent can occupy}}
%{{structureOfAC.png | Structure of an Agent Controller}}
%{{structureOfACNetwork.png | Structure of an Agent Controller Network}}
%{{flowOfMessages.png | Flow of Messages}}
%{{highLevelArch.png | High Level Architecture of Phoenix}}
